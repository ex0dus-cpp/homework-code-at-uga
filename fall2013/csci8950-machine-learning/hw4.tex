\documentclass{article}

\usepackage{fancyhdr} % Required for custom headers
\usepackage{lastpage} % Required to determine the last page for the footer
\usepackage{extramarks} % Required for headers and footers
\usepackage[usenames,dvipsnames]{color} % Required for custom colors
\usepackage{graphicx} % Required to insert images
\usepackage{listings} % Required for insertion of code
\usepackage{courier} % Required for the courier font
\usepackage{lipsum} % Used for inserting dummy 'Lorem ipsum' text into the template
\usepackage{amssymb}
\usepackage{tikz}
\usepackage{hyperref}

% Margins
\topmargin=-0.45in
\evensidemargin=0in
\oddsidemargin=0in
\textwidth=6.5in
\textheight=9.0in
\headsep=0.25in

\linespread{1.1} % Line spacing

% Set up the header and footer
\pagestyle{fancy}
\lhead{\hmwkAuthorName} % Top left header
\chead{\hmwkClass\ (\hmwkClassInstructor\ \hmwkClassTime): \hmwkTitle} % Top center head
\rhead{\firstxmark} % Top right header
\lfoot{\lastxmark} % Bottom left footer
\cfoot{} % Bottom center footer
\rfoot{Page\ \thepage\ of\ \protect\pageref{LastPage}} % Bottom right footer
\renewcommand\headrulewidth{0.4pt} % Size of the header rule
\renewcommand\footrulewidth{0.4pt} % Size of the footer rule

\setlength\parindent{0pt} % Removes all indentation from paragraphs

\definecolor{MyDarkGreen}{rgb}{0.0,0.4,0.0} % This is the color used for comments
\lstloadlanguages{Perl} % Load Perl syntax for listings, for a list of other languages supported see: ftp://ftp.tex.ac.uk/tex-archive/macros/latex/contrib/listings/listings.pdf
\lstset{language=Perl, % Use Perl in this example
        frame=single, % Single frame around code
        basicstyle=\small\ttfamily, % Use small true type font
        keywordstyle=[1]\color{Blue}\bf, % Perl functions bold and blue
        keywordstyle=[2]\color{Purple}, % Perl function arguments purple
        keywordstyle=[3]\color{Blue}\underbar, % Custom functions underlined and blue
        identifierstyle=, % Nothing special about identifiers
        commentstyle=\usefont{T1}{pcr}{m}{sl}\color{MyDarkGreen}\small, % Comments small dark green courier font
        stringstyle=\color{Purple}, % Strings are purple
        showstringspaces=false, % Don't put marks in string spaces
        tabsize=5, % 5 spaces per tab
        %
        % Put standard Perl functions not included in the default language here
        morekeywords={rand},
        %
        % Put Perl function parameters here
        morekeywords=[2]{on, off, interp},
        %
        % Put user defined functions here
        morekeywords=[3]{test},
       	%
        morecomment=[l][\color{Blue}]{...}, % Line continuation (...) like blue comment
        numbers=left, % Line numbers on left
        firstnumber=1, % Line numbers start with line 1
        numberstyle=\tiny\color{Blue}, % Line numbers are blue and small
        stepnumber=5 % Line numbers go in steps of 5
}

% Creates a new command to include a perl script, the first parameter is the filename of the script (without .pl), the second parameter is the caption
\newcommand{\pythonscript}[2]{
\begin{itemize}
\item[]\lstinputlisting[caption=#2,label=#1,language=python]{#1.py}
\end{itemize}
}

%----------------------------------------------------------------------------------------
%	DOCUMENT STRUCTURE COMMANDS
%	Skip this unless you know what you're doing
%----------------------------------------------------------------------------------------

% Header and footer for when a page split occurs within a problem environment
\newcommand{\enterProblemHeader}[1]{
\nobreak\extramarks{#1}{#1 continued on next page\ldots}\nobreak
\nobreak\extramarks{#1 (continued)}{#1 continued on next page\ldots}\nobreak
}

% Header and footer for when a page split occurs between problem environments
\newcommand{\exitProblemHeader}[1]{
\nobreak\extramarks{#1 (continued)}{#1 continued on next page\ldots}\nobreak
\nobreak\extramarks{#1}{}\nobreak
}

\setcounter{secnumdepth}{0} % Removes default section numbers
\newcounter{homeworkProblemCounter} % Creates a counter to keep track of the number of problems

\newcommand{\homeworkProblemName}{}
\newenvironment{homeworkProblem}[1][Problem \arabic{homeworkProblemCounter}]{ % Makes a new environment called homeworkProblem which takes 1 argument (custom name) but the default is "Problem #"
\stepcounter{homeworkProblemCounter} % Increase counter for number of problems
\renewcommand{\homeworkProblemName}{#1} % Assign \homeworkProblemName the name of the problem
\section{\homeworkProblemName} % Make a section in the document with the custom problem count
\enterProblemHeader{\homeworkProblemName} % Header and footer within the environment
}{
\exitProblemHeader{\homeworkProblemName} % Header and footer after the environment
}

\newcommand{\problemAnswer}[1]{ % Defines the problem answer command with the content as the only argument
    \noindent\framebox[\columnwidth][c]{
        \begin{minipage}{0.98\columnwidth}
            \textbf{Answer}

            #1
        \end{minipage}
    } % Makes the box around the problem answer and puts the content inside
}

\newcommand{\homeworkSectionName}{}
\newenvironment{homeworkSection}[1]{ % New environment for sections within homework problems, takes 1 argument - the name of the section
\renewcommand{\homeworkSectionName}{#1} % Assign \homeworkSectionName to the name of the section from the environment argument
\subsection{\homeworkSectionName} % Make a subsection with the custom name of the subsection
\enterProblemHeader{\homeworkProblemName\ [\homeworkSectionName]} % Header and footer within the environment
}{
\enterProblemHeader{\homeworkProblemName} % Header and footer after the environment
}

%----------------------------------------------------------------------------------------
%	NAME AND CLASS SECTION
%----------------------------------------------------------------------------------------

\newcommand{\hmwkTitle}{Assignment \#4} % Assignment title
\newcommand{\hmwkDueDate}{Tuesday, Oct 17, 2013} % Due date
\newcommand{\hmwkClass}{CSCI 8950} % Course/class
\newcommand{\hmwkClassTime}{} % Class/lecture time
\newcommand{\hmwkClassInstructor}{Dr. Rasheed} % Teacher/lecturer
\newcommand{\hmwkAuthorName}{Yuchen Ying} % Your name

%----------------------------------------------------------------------------------------
%	TITLE PAGE
%----------------------------------------------------------------------------------------

\title{
\vspace{2in}
\textmd{\textbf{\hmwkClass:\ \hmwkTitle}}\\
\normalsize\vspace{0.1in}\small{Due\ on\ \hmwkDueDate}\\
\vspace{0.1in}\large{\textit{\hmwkClassInstructor\ \hmwkClassTime}}
\vspace{3in}
}

\author{\textbf{\hmwkAuthorName}}
\date{} % Insert date here if you want it to appear below your name

%----------------------------------------------------------------------------------------

\begin{document}

\maketitle

%----------------------------------------------------------------------------------------
%	TABLE OF CONTENTS
%----------------------------------------------------------------------------------------

%\setcounter{tocdepth}{1} % Uncomment this line if you don't want subsections listed in the ToC

\newpage
\tableofcontents
\newpage

%----------------------------------------------------------------------------------------
%	PROBLEM 1
%----------------------------------------------------------------------------------------

% To have just one problem per page, simply put a \clearpage after each problem

\begin{homeworkProblem}
    For this assignment you need to create or use a suitable \texttt{Backpropagation} neural network package.

    You should \textbf{experiment} to find the best number of hidden units and report your experimental results

    \problemAnswer{
        The dataset I used was \textbf{Tic-Tac-Toe Endgame Data Set} (\url{http://archive.ics.uci.edu/ml/datasets/Tic-Tac-Toe+Endgame})

        For the experiment, I write Code List \ref{code/hw4/experiment} (at the end of this document) to do the experiment. You need to specify a number as the first argument of this Python script, and the code will run 500 iteration on a Feed Forward Neural Network, using the specified number of hidden units, then do a 2-fold cross-validation and calculate the MSE (mean square error) as a proof of accuracy.

        The result of the experiment can be found at \ref{table_1} (at the end of this document).  To have a simpler NN as well as higher accuracy, I'll use 3 hidden units.

        The design of my neural network would be:

        \begin{itemize}
            \item Input Layer: 9 units corresponding to 9 features in input dataset
            \item Hidden Layer: 3 units as the experiment suggest. Use Sigmoid function to squash data from input layer
            \item Output Layer: 1 units indicating the positive/negative output, where 1 indicates positive and 0 indicates negative
        \end{itemize}

        Parameters I used in my \texttt{Backpropagation} trainer:

        \begin{itemize}
            \item $weightdecay = 0.001$
            \item $learningrate = 0.1$
            \item $momentum = 0.0$ (default value)
            \item $lrdecay = 1.0$ (default value)
        \end{itemize}

        The main code I used to do the learning is Code List \ref{code/hw4/main} (at the end of this document).

        It will split the raw data into two parts, 75\% of them are used as training set, 25\% of them are used as validation set.

        It then iterate using \texttt{backpropagation} for 6000 times. Every time it find a minimum error, it will do 10 more iteration and see if it's really the minimum error.

        The final training error: 0.088998. The corresponding validation error: 0.107706

        After training, I did a 10-fold cross-validation over the whole dataset, and the MSE (mean squared error) is 0.178744789109

        Total training time is 83m23.387s, with 6000 iteration.
    }


\end{homeworkProblem}

\clearpage
\pythonscript{code/hw4/experiment}{Python code to do experiment}

\clearpage
\pythonscript{code/hw4/main}{Main Python code}

\clearpage

\begin{table}[hbp]
    \centering
    \begin{tabular}{|c|c|}
        \hline
        Count of Hidden Units & Mean Squared Error \\
        \hline
        1 & 1.09002171863 \\
        \hline
        2 & 2.73573239857 \\
        \hline
        3 & 0.793887549803 \\
        \hline
        4 & 15.2119612277 \\
        \hline
        5 & 6.59174665629 \\
        \hline
        6 & 12.7988201822 \\
        \hline
        7 & 1.91940534104 \\
        \hline
        8 & 0.755597260501 \\
        \hline
        9 & 12.442981528 \\
        \hline
    \end{tabular}
    \caption{2-fold cross-validation result with different count of hidden units}
    \label{table_1}
\end{table}
\end{document}
