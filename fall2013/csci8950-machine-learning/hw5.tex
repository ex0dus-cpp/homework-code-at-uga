\documentclass{article}

\usepackage{fancyhdr} % Required for custom headers
\usepackage{lastpage} % Required to determine the last page for the footer
\usepackage{extramarks} % Required for headers and footers
\usepackage[usenames,dvipsnames]{color} % Required for custom colors
\usepackage{graphicx} % Required to insert images
\usepackage{listings} % Required for insertion of code
\usepackage{courier} % Required for the courier font
\usepackage{lipsum} % Used for inserting dummy 'Lorem ipsum' text into the template
\usepackage{amssymb}
\usepackage{tikz}
\usepackage{hyperref}

% Margins
\topmargin=-0.45in
\evensidemargin=0in
\oddsidemargin=0in
\textwidth=6.5in
\textheight=9.0in
\headsep=0.25in

\linespread{1.1} % Line spacing

% Set up the header and footer
\pagestyle{fancy}
\lhead{\hmwkAuthorName} % Top left header
\chead{\hmwkClass\ (\hmwkClassInstructor\ \hmwkClassTime): \hmwkTitle} % Top center head
\rhead{\firstxmark} % Top right header
\lfoot{\lastxmark} % Bottom left footer
\cfoot{} % Bottom center footer
\rfoot{Page\ \thepage\ of\ \protect\pageref{LastPage}} % Bottom right footer
\renewcommand\headrulewidth{0.4pt} % Size of the header rule
\renewcommand\footrulewidth{0.4pt} % Size of the footer rule

\setlength\parindent{0pt} % Removes all indentation from paragraphs

\definecolor{MyDarkGreen}{rgb}{0.0,0.4,0.0} % This is the color used for comments
\lstloadlanguages{Perl} % Load Perl syntax for listings, for a list of other languages supported see: ftp://ftp.tex.ac.uk/tex-archive/macros/latex/contrib/listings/listings.pdf
\lstset{language=Perl, % Use Perl in this example
        frame=single, % Single frame around code
        basicstyle=\small\ttfamily, % Use small true type font
        keywordstyle=[1]\color{Blue}\bf, % Perl functions bold and blue
        keywordstyle=[2]\color{Purple}, % Perl function arguments purple
        keywordstyle=[3]\color{Blue}\underbar, % Custom functions underlined and blue
        identifierstyle=, % Nothing special about identifiers
        commentstyle=\usefont{T1}{pcr}{m}{sl}\color{MyDarkGreen}\small, % Comments small dark green courier font
        stringstyle=\color{Purple}, % Strings are purple
        showstringspaces=false, % Don't put marks in string spaces
        tabsize=5, % 5 spaces per tab
        %
        % Put standard Perl functions not included in the default language here
        morekeywords={rand},
        %
        % Put Perl function parameters here
        morekeywords=[2]{on, off, interp},
        %
        % Put user defined functions here
        morekeywords=[3]{test},
       	%
        morecomment=[l][\color{Blue}]{...}, % Line continuation (...) like blue comment
        numbers=left, % Line numbers on left
        firstnumber=1, % Line numbers start with line 1
        numberstyle=\tiny\color{Blue}, % Line numbers are blue and small
        stepnumber=5 % Line numbers go in steps of 5
}

% Creates a new command to include a perl script, the first parameter is the filename of the script (without .pl), the second parameter is the caption
\newcommand{\pythonscript}[2]{
\begin{itemize}
\item[]\lstinputlisting[caption=#2,label=#1,language=python]{#1.py}
\end{itemize}
}

%----------------------------------------------------------------------------------------
%	DOCUMENT STRUCTURE COMMANDS
%	Skip this unless you know what you're doing
%----------------------------------------------------------------------------------------

% Header and footer for when a page split occurs within a problem environment
\newcommand{\enterProblemHeader}[1]{
\nobreak\extramarks{#1}{#1 continued on next page\ldots}\nobreak
\nobreak\extramarks{#1 (continued)}{#1 continued on next page\ldots}\nobreak
}

% Header and footer for when a page split occurs between problem environments
\newcommand{\exitProblemHeader}[1]{
\nobreak\extramarks{#1 (continued)}{#1 continued on next page\ldots}\nobreak
\nobreak\extramarks{#1}{}\nobreak
}

\setcounter{secnumdepth}{0} % Removes default section numbers
\newcounter{homeworkProblemCounter} % Creates a counter to keep track of the number of problems

\newcommand{\homeworkProblemName}{}
\newenvironment{homeworkProblem}[1][Problem \arabic{homeworkProblemCounter}]{ % Makes a new environment called homeworkProblem which takes 1 argument (custom name) but the default is "Problem #"
\stepcounter{homeworkProblemCounter} % Increase counter for number of problems
\renewcommand{\homeworkProblemName}{#1} % Assign \homeworkProblemName the name of the problem
\section{\homeworkProblemName} % Make a section in the document with the custom problem count
\enterProblemHeader{\homeworkProblemName} % Header and footer within the environment
}{
\exitProblemHeader{\homeworkProblemName} % Header and footer after the environment
}

\newcommand{\problemAnswer}[1]{ % Defines the problem answer command with the content as the only argument
    \noindent\framebox[\columnwidth][c]{
        \begin{minipage}{0.98\columnwidth}
            \textbf{Answer}

            #1
        \end{minipage}
    } % Makes the box around the problem answer and puts the content inside
}

\newcommand{\homeworkSectionName}{}
\newenvironment{homeworkSection}[1]{ % New environment for sections within homework problems, takes 1 argument - the name of the section
\renewcommand{\homeworkSectionName}{#1} % Assign \homeworkSectionName to the name of the section from the environment argument
\subsection{\homeworkSectionName} % Make a subsection with the custom name of the subsection
\enterProblemHeader{\homeworkProblemName\ [\homeworkSectionName]} % Header and footer within the environment
}{
\enterProblemHeader{\homeworkProblemName} % Header and footer after the environment
}

%----------------------------------------------------------------------------------------
%	NAME AND CLASS SECTION
%----------------------------------------------------------------------------------------

\newcommand{\hmwkTitle}{Assignment \#5} % Assignment title
\newcommand{\hmwkDueDate}{Tuesday, Nov 12, 2013} % Due date
\newcommand{\hmwkClass}{CSCI 8950} % Course/class
\newcommand{\hmwkClassTime}{} % Class/lecture time
\newcommand{\hmwkClassInstructor}{Dr. Rasheed} % Teacher/lecturer
\newcommand{\hmwkAuthorName}{Yuchen Ying} % Your name

%----------------------------------------------------------------------------------------
%	TITLE PAGE
%----------------------------------------------------------------------------------------

\title{
\vspace{2in}
\textmd{\textbf{\hmwkClass:\ \hmwkTitle}}\\
\normalsize\vspace{0.1in}\small{Due\ on\ \hmwkDueDate}\\
\vspace{0.1in}\large{\textit{\hmwkClassInstructor\ \hmwkClassTime}}
\vspace{3in}
}

\author{\textbf{\hmwkAuthorName}}
\date{} % Insert date here if you want it to appear below your name

%----------------------------------------------------------------------------------------

\begin{document}

\maketitle

%----------------------------------------------------------------------------------------
%	TABLE OF CONTENTS
%----------------------------------------------------------------------------------------

%\setcounter{tocdepth}{1} % Uncomment this line if you don't want subsections listed in the ToC

\newpage
\tableofcontents
\newpage

%----------------------------------------------------------------------------------------
%	PROBLEM 1
%----------------------------------------------------------------------------------------

% To have just one problem per page, simply put a \clearpage after each problem

\begin{homeworkProblem}
    Consider the following training set of samples for machine learning:

    \begin{table}[hbp]
        \centering
        \begin{tabular}{|c|c|c|c|c|c|}
            \hline
            Example & A1 & A2 & A3 & A4 & label \\
            \hline
            1 & 1 & 2 & 2 & 2 & + \\
            2 & 1 & 1 & 1 & 1 & - \\
            3 & 2 & 3 & 2 & 1 & + \\
            4 & 1 & 3 & 3 & 3 & - \\
            5 & 3 & 1 & 2 & 1 & + \\
            6 & 1 & 1 & 1 & 2 & + \\
            \hline
        \end{tabular}
        %\caption{}
        \label{table_1}
    \end{table}

    The attributes \textbf{A1} through \textbf{A4} are integers with values in the range $[1,2,3]$ each.

    (a) Give a classifier in the GIL format that can correctly classify all the training examples.

    \problemAnswer{
        %$(1,2,2,2) : 1$

        %$(2,3,2,1) : 1$

        %$(3,1,2,1) : 1$

        %$(1,1,1,2) : 1$

        %$(1,1,1,1) : 0$

        %$(1,3,3,3) : 0$

        %$<001|010|010|010> : 1$

        %$<010|100|010|001> : 1$

        %$<100|001|010|001> : 1$

        %$<001|001|001|010> : 1$

        %$<001|001|001|001> : 0$

        %$<001|100|100|100> : 0$

        $(<A3=2>) \vee (<A3=1 \wedge A4=2>)$

        $<111|111|010|111 \vee 111|111|001|010>$
    }

    (b) How would the GIL classifier in Part (a) above classify the following examples: $(1,2,2,3)$ and $(3,2,1,1)$

    \problemAnswer{
        $(1,2,2,3) \rightarrow +$

        $(3,2,1,1) \rightarrow -$
    }

    (c) Of the following two methods, which do you think will be a more \textbf{efficient} learning method for this problem and why?

    \begin{itemize}
        \item Using the GIL classifier system learning
        \item Using ID3 to learn a decision tree
    \end{itemize}

    \problemAnswer{
        Decision tree would be more efficient. There are only 4 attribute and the tree has at most 4 layers without leaf, which is much faster than GA iteration.
    }
\end{homeworkProblem}

\begin{homeworkProblem}
    (a) Propose a lazy version of the back-propagation algorithm for training neural networks. What are the advantages and disadvantages of your algorithm, compared to the original back-propagation algorithm?
    \problemAnswer{
        Modify the weight of each node with fixed portion of original value ($w \gets \pm \eta w$ where $\eta$ is a small value), instead of trying to minimize the squared error between output values and target values.

        It's obviously much faster than the normal back-propagation. It may take more time to learn the correct concept because the weight update no longer towards the steepest descent on the error surface.
    }

    (b) Propose an eager version of the nearest neighbour algorithm for classification. What are the advantages and disadvantages of your algorithm, compared to the original nearest neighbour algorithm?

    \problemAnswer{
        The kd-tree indexing for k-nearest neighbor algorithm.

        It will take more time in training because each instance in training set requires additional calculation to put it in the proper location of a tree structure. It will be faster since looking for nearest neighbor can be done in $O(log(n))$ time.
    }
\end{homeworkProblem}

\begin{homeworkProblem}
    Consider the following diagram of a set of 8 instances for machine learning:

    \begin{tikzpicture}[thick, scale=1.5, every node/.style={scale=1.5}]
        % Draw axes
        \draw [<->,thick] (0,5) node (yaxis) [above] {$X1$}
        |- (6,0) node (xaxis) [right] {$X2$};
        %\draw[style=help lines] (0,0) grid (10.5, 2.5);
        \foreach \x in {0,1,...,6} \draw (\x cm, 0 cm) node[anchor=north]{\x};
        \foreach \y in {0,1,...,5}  \draw (0 cm, \y cm) node[anchor=east]{\y};

        \node (1) at (3,1) {*};
        \node (2) at (4,1) {*};
        \node (3) at (5,2) {*};
        \node (4) at (5,3) {*};
        \node (5) at (4,4) {*};
        \node (6) at (3,4) {*};
        \node (7) at (2,3) {*};
        \node (8) at (2,2) {*};
    \end{tikzpicture}

    (a) Consider the hypothesis space $H1$ consisting of all possible \textbf{circles} in the plane (i.e. each hypothesis $h$ in $H1$ is a circle which classifies all points in it as positive and all points outside it as negative). Does $H1$ shatter the given set of instances? Briefly justify your answer.

    \problemAnswer{
        No.
    }

    (b) Consider the hypothesis space $H2$ consisting of all possible \textbf{rectangles} in the plane. Does $H2$ shatter the given set of instances? Briefly justify your answer.

    \problemAnswer{
        No.
    }

    (c) Based \textbf{only} on your answers to parts (a) and (b) above, what can you conclude about the VC dimensions of $H1$ and $H2$

    \problemAnswer{
    }
\end{homeworkProblem}
\end{document}
